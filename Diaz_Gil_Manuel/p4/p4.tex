\documentclass[]{article}

\usepackage[left=2.00cm, right=2.00cm, top=2.00cm, bottom=2.00cm]{geometry}
\usepackage[spanish,es-noshorthands]{babel}
\usepackage[utf8]{inputenc} % para tildes y ñ

%opening
\title{Práctica 4. Exploración de grafos}
\author{Manuel Diaz Gil \\ % mantenga las dos barras al final de la línea y este comentario
manuel.diazgil@alum.uca.es \\ % mantenga las dos barras al final de la línea y este comentario
Teléfono: 667361517 \\ % mantenga las dos barras al final de la linea y este comentario
NIF: 45382945N \\ % mantenga las dos barras al final de la línea y este comentario
}


\begin{document}

\maketitle

%\begin{abstract}
%\end{abstract}

% Ejemplo de ecuación a trozos
%
%$f(i,j)=\left\{ 
%  \begin{array}{lcr}
%      i + j & si & i < j \\ % caso 1
%      i + 7 & si & i = 1 \\ % caso 2
%      2 & si & i \geq j     % caso 3
%  \end{array}
%\right.$

\begin{enumerate}
\item Comente el funcionamiento del algoritmo y describa las estructuras necesarias para llevar a cabo su implementación.

Algoritmo usado A* que calcula el mejor camino para el uco segun sus posibilidades, recoriendo los nodos y expandiendose segun su valor.

Estructuras de datos usadas monticulo para ordenar el mejor nodo y set para guardar los nodos abiertos y usados, y asi poder comprobar cuales se han recorrido ya y cuales quedan por recorrer. 


\item Incluya a continuación el código fuente relevante del algoritmo.

\begin{lstlisting}
void ordenacionInsercion(std::vector<celda>& c, int i, int k){
    int j;
    celda aux;
    for(i = 0;i<k;i++){
        aux = c[i];
        for(j = i;j>0 && (aux<c[j-1]);j--){
                c[j]=c[j-1];
        }
        c[j]=aux;
    }
}

void fusion(std::vector<celda>& c,int i,int k,int j){
    int n = j-i;
    std::vector<celda> aux;
    int iter=i;
    int iter2=k;
    for (int l = 0; l < n; ++l)
    {
        if(iter<=k && (iter2>j-1 || c[iter]<=c[iter2])){
            aux.push_back(c[iter]);
            ++iter;
        }
        else{
            aux.push_back(c[iter2]);
            ++iter2;
        }       
    }
    for (int l = 0; l < n; ++l)
    {
        c[l+i]=aux[l];
    }
}

void ordenacionFusion(std::vector<celda>& c,int i, int j){
    int n = j-i;
    if (n<3){
        ordenacionInsercion(c,i,j);
    }
    else{
        int k=j-((j-i)/2);
        ordenacionFusion(c,i,k);
        ordenacionFusion(c,k,j);
        fusion(c,i,k,j);    
    }

}
\end{lstlisting}


\end{enumerate}

Todo el material incluido en esta memoria y en los ficheros asociados es de mi autoría o ha sido facilitado por los profesores de la asignatura. Haciendo entrega de esta práctica confirmo que he leído la normativa de la asignatura, incluido el punto que respecta al uso de material no original.

\end{document}
