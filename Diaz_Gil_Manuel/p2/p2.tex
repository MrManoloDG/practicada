\documentclass[]{article}

\usepackage[left=2.00cm, right=2.00cm, top=2.00cm, bottom=2.00cm]{geometry}
\usepackage[spanish,es-noshorthands]{babel}
\usepackage[utf8]{inputenc} % para tildes y ñ
\usepackage{graphicx} % para las figuras
\usepackage{xcolor}
\usepackage{listings} % para el código fuente en c++

\lstdefinestyle{customc}{
  belowcaptionskip=1\baselineskip,
  breaklines=true,
  frame=single,
  xleftmargin=\parindent,
  language=C++,
  showstringspaces=false,
  basicstyle=\footnotesize\ttfamily,
  keywordstyle=\bfseries\color{green!40!black},
  commentstyle=\itshape\color{gray!40!gray},
  identifierstyle=\color{black},
  stringstyle=\color{orange},
}
\lstset{style=customc}


%opening
\title{Práctica 2. Programación dinámica}
\author{Manuel Diaz Gil \\ % mantenga las dos barras al final de la línea y este comentario
manuel.diazgil@alum.uca.es \\ % mantenga las dos barras al final de la línea y este comentario
Teléfono: 667361517 \\ % mantenga las dos barras al final de la linea y este comentario
NIF: 45382945N \\ % mantenga las dos barras al final de la línea y este comentario
}


\begin{document}

\maketitle

%\begin{abstract}
%\end{abstract}

% Ejemplo de ecuación a trozos
%
%$f(i,j)=\left\{ 
%  \begin{array}{lcr}
%      i + j & si & i < j \\ % caso 1
%      i + 7 & si & i = 1 \\ % caso 2
%      2 & si & i \geq j     % caso 3
%  \end{array}
%\right.$

\begin{enumerate}
\item Formalice a continuación y describa la función que asigna un determinado valor a cada uno de los tipos de defensas.

Algoritmo usado A* que calcula el mejor camino para el uco segun sus posibilidades, recoriendo los nodos y expandiendose segun su valor.

Estructuras de datos usadas monticulo para ordenar el mejor nodo y set para guardar los nodos abiertos y usados, y asi poder comprobar cuales se han recorrido ya y cuales quedan por recorrer. 


\item Describa la estructura o estructuras necesarias para representar la tabla de subproblemas resueltos.

\begin{lstlisting}
void ordenacionInsercion(std::vector<celda>& c, int i, int k){
    int j;
    celda aux;
    for(i = 0;i<k;i++){
        aux = c[i];
        for(j = i;j>0 && (aux<c[j-1]);j--){
                c[j]=c[j-1];
        }
        c[j]=aux;
    }
}

void fusion(std::vector<celda>& c,int i,int k,int j){
    int n = j-i;
    std::vector<celda> aux;
    int iter=i;
    int iter2=k;
    for (int l = 0; l < n; ++l)
    {
        if(iter<=k && (iter2>j-1 || c[iter]<=c[iter2])){
            aux.push_back(c[iter]);
            ++iter;
        }
        else{
            aux.push_back(c[iter2]);
            ++iter2;
        }       
    }
    for (int l = 0; l < n; ++l)
    {
        c[l+i]=aux[l];
    }
}

void ordenacionFusion(std::vector<celda>& c,int i, int j){
    int n = j-i;
    if (n<3){
        ordenacionInsercion(c,i,j);
    }
    else{
        int k=j-((j-i)/2);
        ordenacionFusion(c,i,k);
        ordenacionFusion(c,k,j);
        fusion(c,i,k,j);    
    }

}
\end{lstlisting}

\item En base a los dos ejercicios anteriores, diseñe un algoritmo que determine el máximo beneficio posible a obtener dada una combinación de defensas y \emph{ases} disponibles. Muestre a continuación el código relevante.

\begin{lstlisting}
// sustituya este codigo por su respuesta
void placeDefenses(...) {

    List<Defense*>::iterator currentDefense = defenses.begin();
    while(currentDefense != defenses.end() && maxAttemps > 0) {

        (*currentDefense)->position.x = ((int)(_RAND2(nCellsWidth))) * cellWidth + cellWidth * 0.5f;
        ...
        ++currentDefense;
    }
}
\end{lstlisting}

\item Diseñe un algoritmo que recupere la combinación óptima de defensas a partir del contenido de la tabla de subproblemas resueltos. Muestre a continuación el código relevante.

\begin{lstlisting}

    it = defenses.begin();
    int currentases = ases;
    selectedIDs.push_back((*it)->id);
    currentases -= (*it)->cost;

    it = defenses.end();
    for (i = defenses.size()-1; i > 0; --i)
    {
        --it;
        if (tabla[i][currentases]!=tabla[i-1][currentases])
        {
            selectedIDs.push_back((*it)->id);
            currentases -= (*it)->cost;
        }
    }
\end{lstlisting}


\end{enumerate}

Todo el material incluido en esta memoria y en los ficheros asociados es de mi autoría o ha sido facilitado por los profesores de la asignatura. Haciendo entrega de este documento confirmo que he leído la normativa de la asignatura, incluido el punto que respecta al uso de material no original.

\end{document}
