\documentclass[]{article}

\usepackage[left=2.00cm, right=2.00cm, top=2.00cm, bottom=2.00cm]{geometry}
\usepackage[spanish,es-noshorthands]{babel}
\usepackage[utf8]{inputenc} % para tildes y ñ
\usepackage{graphicx} % para las figuras
\usepackage{xcolor}
\usepackage{listings} % para el código fuente en c++

\lstdefinestyle{customc}{
  belowcaptionskip=1\baselineskip,
  breaklines=true,
  frame=single,
  xleftmargin=\parindent,
  language=C++,
  showstringspaces=false,
  basicstyle=\footnotesize\ttfamily,
  keywordstyle=\bfseries\color{green!40!black},
  commentstyle=\itshape\color{gray!40!gray},
  identifierstyle=\color{black},
  stringstyle=\color{orange},
}
\lstset{style=customc}


%opening
\title{Práctica 3. Divide y vencerás}
\author{Manuel Diaz Gil \\ % mantenga las dos barras al final de la línea y este comentario
manuel.diazgil@alum.uca.es \\ % mantenga las dos barras al final de la línea y este comentario
Teléfono: 667361517 \\ % mantenga las dos barras al final de la linea y este comentario
NIF: 45382945N \\ % mantenga las dos barras al final de la línea y este comentario
}


\begin{document}

\maketitle

%\begin{abstract}
%\end{abstract}

% Ejemplo de ecuación a trozos
%
%$f(i,j)=\left\{ 
%  \begin{array}{lcr}
%      i + j & si & i < j \\ % caso 1
%      i + 7 & si & i = 1 \\ % caso 2
%      2 & si & i \geq j     % caso 3
%  \end{array}
%\right.$

\begin{enumerate}
\item Describa las estructuras de datos utilizados en cada caso para la representación del terreno de batalla. 

Algoritmo usado A* que calcula el mejor camino para el uco segun sus posibilidades, recoriendo los nodos y expandiendose segun su valor.

Estructuras de datos usadas monticulo para ordenar el mejor nodo y set para guardar los nodos abiertos y usados, y asi poder comprobar cuales se han recorrido ya y cuales quedan por recorrer. 


\item Implemente su propia versión del algoritmo de ordenación por fusión. Muestre a continuación el código fuente relevante. 

\begin{lstlisting}
void ordenacionInsercion(std::vector<celda>& c, int i, int k){
    int j;
    celda aux;
    for(i = 0;i<k;i++){
        aux = c[i];
        for(j = i;j>0 && (aux<c[j-1]);j--){
                c[j]=c[j-1];
        }
        c[j]=aux;
    }
}

void fusion(std::vector<celda>& c,int i,int k,int j){
    int n = j-i;
    std::vector<celda> aux;
    int iter=i;
    int iter2=k;
    for (int l = 0; l < n; ++l)
    {
        if(iter<=k && (iter2>j-1 || c[iter]<=c[iter2])){
            aux.push_back(c[iter]);
            ++iter;
        }
        else{
            aux.push_back(c[iter2]);
            ++iter2;
        }       
    }
    for (int l = 0; l < n; ++l)
    {
        c[l+i]=aux[l];
    }
}

void ordenacionFusion(std::vector<celda>& c,int i, int j){
    int n = j-i;
    if (n<3){
        ordenacionInsercion(c,i,j);
    }
    else{
        int k=j-((j-i)/2);
        ordenacionFusion(c,i,k);
        ordenacionFusion(c,k,j);
        fusion(c,i,k,j);    
    }

}
\end{lstlisting}


\item Implemente su propia versión del algoritmo de ordenación rápida. Muestre a continuación el código fuente relevante. 

\begin{lstlisting}
// sustituya este codigo por su respuesta
void placeDefenses(...) {

    List<Defense*>::iterator currentDefense = defenses.begin();
    while(currentDefense != defenses.end() && maxAttemps > 0) {

        (*currentDefense)->position.x = ((int)(_RAND2(nCellsWidth))) * cellWidth + cellWidth * 0.5f;
        ...
        ++currentDefense;
    }
}
\end{lstlisting}

\item Realice pruebas de caja negra para asegurar el correcto funcionamiento de los algoritmos de ordenación implementados en los ejercicios anteriores. Detalle a continuación el código relevante.

\begin{lstlisting}

    it = defenses.begin();
    int currentases = ases;
    selectedIDs.push_back((*it)->id);
    currentases -= (*it)->cost;

    it = defenses.end();
    for (i = defenses.size()-1; i > 0; --i)
    {
        --it;
        if (tabla[i][currentases]!=tabla[i-1][currentases])
        {
            selectedIDs.push_back((*it)->id);
            currentases -= (*it)->cost;
        }
    }
\end{lstlisting}


\item Analice de forma teórica la complejidad de las diferentes versiones del algoritmo de colocación de defensas en función de la estructura de representación del terreno de batalla elegida. Comente a continuación los resultados. Suponga un terreno de batalla cuadrado en todos los casos. 

Escriba aquí su respuesta al ejercicio 5.

\item Incluya a continuación una gráfica con los resultados obtenidos. Utilice un esquema indirecto de medida (considere un error absoluto de valor 0.01 y un error relativo de valor 0.001). Considere en su análisis los planetas con códigos 1500, 2500, 3500,..., 10500. Incluya en el análisis los planetas que considere oportunos para mostrar información relevante.

\begin{lstlisting}
std::priority_queue<celda> mceldas2;
    for (int i = 0; i < nCellsWidth; ++i)
    {
    	for (int j = 0; j < nCellsHeight; ++j)
    	{
    		mceldas2.push(celda(i,j,cellValue2(i,j,freeCells,nCellsWidth,nCellsHeight,mapWidth,mapHeight,obstacles,defenses)));
    	}
    }
    

    while(currentDefense != defenses.end()) {
    	colocado=false;
    	while(!mceldas2.empty() && !colocado){
    		cactual=mceldas2.top();
            mceldas2.pop();
    		if(factible(cactual.row,cactual.col,nCellsWidth,nCellsHeight,mapWidth,mapHeight,obstacles,defenses,currentDefense)){
    			(*currentDefense)->position.x = (cactual.row * cellWidth) + cellWidth * 0.5f;
        		(*currentDefense)->position.y = (cactual.col * cellHeight) + cellHeight * 0.5f;
        		(*currentDefense)->position.z = 0; 
                colocado=true;
    		}
    	}
        ++currentDefense;
    	
    }

#ifdef PRINT_DEFENSE_STRATEGY

    float** cellValues = new float* [nCellsHeight]; 
    for(int i = 0; i < nCellsHeight; ++i) {
       cellValues[i] = new float[nCellsWidth];
       for(int j = 0; j < nCellsWidth; ++j) {
           cellValues[i][j] = ((int)(cellValue(i, j))) % 256;
       }
    }
    dPrintMap("strategy.ppm", nCellsHeight, nCellsWidth, cellHeight, cellWidth, freeCells
                         , cellValues, std::list<Defense*>(), true);

    for(int i = 0; i < nCellsHeight ; ++i)
        delete [] cellValues[i];
	delete [] cellValues;
	cellValues = NULL;

\end{lstlisting}


\end{enumerate}

Todo el material incluido en esta memoria y en los ficheros asociados es de mi autoría o ha sido facilitado por los profesores de la asignatura. Haciendo entrega de este documento confirmo que he leído la normativa de la asignatura, incluido el punto que respecta al uso de material no original.

\end{document}
